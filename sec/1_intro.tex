\section{Introduction}
\label{sec:intro}

% Why are we tracking cells in the first place (Cellular Activity Monitoring)

Monitoring cells and their activities underpin experiments across the biological sciences.  Various types of microscopes can be employed to produce videos or time-sequence observations of 3d volumes, allowing them to collect information about how cells interact with each other and external stimuli.  This can look like light-sheet, fluorescent, or confocal microscopy, each requiring tracking the cells in the volume they occupy; fully automated tracking of sufficient quality at scale has not yet been perfected~\cite{mavska2023cell}.\\

The first candidate for direct application from the rest of the computer vision community would be solutions to multi-object tracking.  In multiobject tracking, the archetypal framing of the problem is one of camera surveillance over people in a volume.  The surveillance camera situation has specific challenges that it has to deal with based on the topic, for instance occlusion where one person passes in front of the other makes tracking more challenging~\cite{luo2021multiple}.  Cells exhibit two specific behaviors of interest distinct from other CV tracking tasks, specifically mitosis and apoptosis. Mitosis is the process by which one cell divides into two identical daughter cells and apoptosis,which is the process of cell death followed by an explosive rupture of the cell. Another major challenge is that almost all multi-object tracking frameworks are appearance-based, learning features that maximize similarity among the same instance across different time steps. This procedure renders less valuable in cell tracking as most cells do not possess such rich appearance distinctiveness~\cite{mavska2023cell}. \\

Researchers in biomedical sciences have taken inspiration from the computer vision community and predominantly used tracking by detection paradigms, where first, the detection or instance segmentation is performed to give individual objects a distinct ID~\cite{bragantini2024ultrack,ershov2022trackmate}. However, most algorithms utilize IOU or Distance-based matching instead of direct appearance-based matching. These values are used as weights in the association matrix to construct a graph. An operation similar to beam search is performed to prune the paths with a lower cost with respect to a predefined threshold.  An integer linear program is used to find the best paths for each initial node in the graph. Constraints are put on the integer linear program to accommodate mitosis (cell division) and cell death. Generally, that is how cell tracking is done~\cite{mavska2023cell}. Some approaches focus on improving the detection paradigm, while others focus on improving their association matrix for the integer linear program~\cite{mavska2023cell}. However, all of these approaches are limited by the datasets they use and are tailored to those. Minimal work has been done to accommodate datasets from various sources. As mentioned previously, these datasets can be collected from many different imaging modalities. Cells can also possess different sizes, shapes, and dynamics. One tracking approach cannot accommodate all of these challenges~\cite{chen2024cmtt}. Therefore, there has to be an adaptive framework that can quickly adapt to a new dataset. However, the adaptation has to be unsupervised and efficient to add real-world constraints.\\

In this work we address the detection paradigm by improving the accuracy of the segmentation stage.  In this case we are adapting the network presented in ~\cite{stringer2021cellpose}, called Cellpose, by adjusting the batch normalization parameters during inference.  Informed by a contrastive self-entropy loss the parameter changes adjust the distribution of the incoming and intermediate data towards the original distribution to acheive better results. We have hypothesized and shown that by performing even one epoch of test time adaptation on cell segmentation will significantly improve cell segmentation performance over not applying Test Time Adaptation (TTA). We will show that our approach of adapting batch normalization based on a novel, unsupervised loss improves performance in knowledge distillation with stochastic restoration. To test our approach we use the Cellpose~\cite{keaton2023celltranspose} model as our base. The model is trained on the Cellpose dataset~\cite{stringer2021cellpose} and then tested on data from TissueNet~\cite{TissueNet,TN2,TN3,TN4}. We compare the Average Precision (AP) and F1 score when the network is applied with and without TTA. 


