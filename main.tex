% CVPR 2024 Paper Template; see https://github.com/cvpr-org/author-kit

\documentclass[10pt,twocolumn,letterpaper]{article}

%%%%%%%%% PAPER TYPE  - PLEASE UPDATE FOR FINAL VERSION
% \usepackage{cvpr}              % To produce the CAMERA-READY version
% \usepackage[review]{cvpr}      % To produce the REVIEW version
\usepackage[pagenumbers]{cvpr} % To force page numbers, e.g. for an arXiv version
\usepackage{lmodern,bm}
% Import additional packages in the preamble file, before hyperref
\input{preamble}

% It is strongly recommended to use hyperref, especially for the review version.
% hyperref with option pagebackref eases the reviewers' job.
% Please disable hyperref *only* if you encounter grave issues, 
% e.g. with the file validation for the camera-ready version.
%
% If you comment hyperref and then uncomment it, you should delete *.aux before re-running LaTeX.
% (Or just hit 'q' on the first LaTeX run, let it finish, and you should be clear).
\definecolor{cvprblue}{rgb}{0.21,0.49,0.74}
\usepackage[pagebackref,breaklinks,colorlinks,citecolor=cvprblue]{hyperref}

%%%%%%%%% PAPER ID  - PLEASE UPDATE
\def\paperID{*****} % *** Enter the Paper ID here
\def\confName{CVPR}
\def\confYear{2025}

%%%%%%%%% TITLE - PLEASE UPDATE


%%%%%%%%% AUTHORS - PLEASE UPDATE
\author{Quinn Jones \qquad Ram Zaveri \\
West Virginia University\\
% Institution1 address\\
{\tt\small \{qjones1, rz0012\}@mix.wvu.edu}
% For a paper whose authors are all at the same institution,
% omit the following lines up until the closing ``}''.
% Additional authors and addresses can be added with ``\and'',
% just like the second author.
% To save space, use either the email address or home page, not both
% \and
% Ram Zaveri\\
% Institution2\\
% First line of institution2 address\\
% {\tt\small secondauthor@i2.org}
}

\begin{document}


\begin{table}[h]
\centering
\begin{tabular}{|c|c|}
\hline
\textbf{Ram} & \textbf{Quinn} \\ \hline
 Prepared Methods and Experiments   & 
 Prepared Introduction, Related Works, and Conclusions
\\ \hline
 &   \\
 \hline
\end{tabular}
\caption{Table for contributions}
\label{tab:ram_quinn}
\end{table}


\pagebreak\pagebreak
\title{Improving Cell Instance Segmentation with Test-Time Adaptation}
\maketitle
\begin{abstract}
Multi Cell Tracking and lineage construction is an ongoing challenge in the biomedical community. Many works have either directly adapted Multi-Object Tracking frameworks or modified them to work for their specific dataset. However, these simple adaptations render two major challenges: (i) individual cells do not possess distinctive appearances as objects would in macro world. (ii) cells go through mitosis and apoptosis. These challenges require developing an approach from ground up. The major challenge within such cases is the variability across cell types and imaging modalities. Even when using approaches tailored to tracking cells, they still render ineffective when distribution changes drastically. Major source of this performance deterioration is the inaccuracies in the detection of the cells even before linking stage. In this work, we propose to alleviate this challenge by introducing a Test-Time Adaptation procedure for the detection phase, which in this case is segmentation. Upon better segmentation, the linking accuracy should also improve, thus increasing the performance of tracking all together in drastically changing distributions of various datasets. 

\end{abstract}    
\section{Introduction}
\label{sec:intro}

% Why are we tracking cells in the first place (Cellular Activity Monitoring)

Monitoring cells and their activities underpin experiments across the biological sciences.  Various types of microscopes can be employed to produce videos or time-sequence observations of 3d volumes, allowing them to collect information about how cells interact with each other and external stimuli.  This can look like light-sheet, fluorescent, or confocal microscopy, each requiring tracking the cells in the volume they occupy; fully automated tracking of sufficient quality at scale has not yet been perfected~\cite{mavska2023cell}.\\

The first candidate for direct application from the rest of the computer vision community would be solutions to multi-object tracking.  In multiobject tracking, the archetypal framing of the problem is one of camera surveillance over people in a volume.  The surveillance camera situation has specific challenges that it has to deal with based on the topic, for instance occlusion where one person passes in front of the other makes tracking more challenging~\cite{luo2021multiple}.  Cells exhibit two specific behaviors of interest distinct from other CV tracking tasks, specifically mitosis and apoptosis. Mitosis is the process by which one cell divides into two identical daughter cells and apoptosis,which is the process of cell death followed by an explosive rupture of the cell. Another major challenge is that almost all multi-object tracking frameworks are appearance-based, learning features that maximize similarity among the same instance across different time steps. This procedure renders less valuable in cell tracking as most cells do not possess such rich appearance distinctiveness~\cite{mavska2023cell}. \\

Researchers in biomedical sciences have taken inspiration from the computer vision community and predominantly used tracking by detection paradigms, where first, the detection or instance segmentation is performed to give individual objects a distinct ID~\cite{bragantini2024ultrack,ershov2022trackmate}. However, most algorithms utilize IOU or Distance-based matching instead of direct appearance-based matching. These values are used as weights in the association matrix to construct a graph. An operation similar to beam search is performed to prune the paths with a lower cost with respect to a predefined threshold.  An integer linear program is used to find the best paths for each initial node in the graph. Constraints are put on the integer linear program to accommodate mitosis (cell division) and cell death. Generally, that is how cell tracking is done~\cite{mavska2023cell}. Some approaches focus on improving the detection paradigm, while others focus on improving their association matrix for the integer linear program~\cite{mavska2023cell}. However, all of these approaches are limited by the datasets they use and are tailored to those. Minimal work has been done to accommodate datasets from various sources. As mentioned previously, these datasets can be collected from many different imaging modalities. Cells can also possess different sizes, shapes, and dynamics. One tracking approach cannot accommodate all of these challenges~\cite{chen2024cmtt}. Therefore, there has to be an adaptive framework that can quickly adapt to a new dataset. However, the adaptation has to be unsupervised and efficient to add real-world constraints.\\

In this work we address the detection paradigm by improving the accuracy of the segmentation stage.  In this case we are adapting the network presented in ~\cite{stringer2021cellpose}, called Cellpose, by adjusting the batch normalization parameters during inference.  Informed by a contrastive self-entropy loss the parameter changes adjust the distribution of the incoming and intermediate data towards the original distribution to acheive better results. We have hypothesized and shown that by performing even one epoch of test time adaptation on cell segmentation will significantly improve cell segmentation performance over not applying Test Time Adaptation (TTA). We will show that our approach of adapting batch normalization based on a novel, unsupervised loss improves performance in knowledge distillation with stochastic restoration. To test our approach we use the Cellpose~\cite{keaton2023celltranspose} model as our base. The model is trained on the Cellpose dataset~\cite{stringer2021cellpose} and then tested on data from TissueNet~\cite{TissueNet,TN2,TN3,TN4}. We compare the Average Precision (AP) and F1 score when the network is applied with and without TTA. 



\section{Related Works}
To discuss the related works we will discuss the challenges of Cell Segmentation and Test-Time Adaptation (TTA).

\subsection{Cell Segmentation}

Cell segmentation is the process of taking images from a microscope and automatically labelling each pixel of the image as belonging to a cell or not (in other words a background pixel).  For the purposes of tracking generally we take this a step further and perform instance segmentation~\cite{INSTANCESEG}.  Instance segmentation adds the extra goal of keeping a seperate label for each of the present 


\begin{itemize}

 

    \item Utrack 
    \item Cellpose
    \item Tissuenet
    
\end{itemize}

\subsection{Test-Time Adaptation}

TTA is a method that attemtpts to solve the issue of drift in the data distribution between the training and testing data. This serves the purpose of making a model that is generalizable which is especially valuable for medical imaging due to its varability. 

In ~\cite{chen2024cmtt,Moshkov2020-uy} recently they apply TTA to cell image tracking/segmentation in different ways.\\

Our approach to test-time adaptation follows in the footsteps of~\cite{Li2018-el}, which introduced batch normalization in test-time adaptation. In~\cite{Li2018-el}, they present \textit{AdaBN}, where they use the property of eliminating covariate shift that BN was originally built for by collecting the batch normalizing statistics of the whole of the target domain before re-classifying with the optimized settings.  This becomes an even more powerful tool when combined with more in-depth criteria for the BN settings.  Such as ~\cite{wang2020tent} which we mirror where they use the Shannon entropy of the model predictions to learn an affine transformation that shifts the output following the normalization.  We improve on it further by adding a simultaneous contrastive flow loss to affect the speed of convergence.\\





\section{Methods}

\subsection{Overview}

Our hypothesis is that performing test time adaptation on cell segmentation will improve cell tracking performance since the associations are based on simple metrics such as IoU and Euclidean distance of the centroids. We propose to pretrain a denoising framework that takes three inputs: The raw image, noisy segmentation from a pretrained segmentation model, and Gaussian splats retrieved from the raw image, and produce a clean segmentation mask as pseudo labels.
Once the pseudo labels are retrieved, we perform knowledge distillation with stochastic restoration of parameters to perform the test-time adaptation. Please refer to \Cref{fig:overall_framework}.


\begin{figure}[t]
    \centering
    \includegraphics[width=9cm]{figs/project_proposal.pdf}
    \caption{Overall Framework.}
    \label{fig:overall_framework}
\end{figure}

\subsection{Cell Segmentation}

To test our hypothesis, we use one of the state-of-the-art cell instance segmentation algorithms as our segmentation framework Cellpose~\cite{stringer2021cellpose}. It is pretrained on diverse datasets and has good generalization capability. However, it is not pretrained on all imaging modalities or cell types, rendering it less useful in out-of-distribution datasets. We propose to utilize their pretrained model and perform test-time adaptation. 

\subsection{Training Gaussian Splats Algorithm}

Radiance fields are a new form of 3d representation which relies on novel differentiable ray-marching, in this talk we will discuss two types Neural Radiance Fields (NeRF) and Gaussian Splatting (GS).   In brief, a radiance field is a learnable representation of a scene, in the case of the NeRFS a large MLP is used as the mechanism and in GS a variable list of ellipsoids.  A raycast is a form of graphics rendering in which the emission of photons onto a pixel is calculated by tracking the reflection of light.  For instance, to find the color of the middle pixel of an image, a ray can be calculated in the opposite direction of how light is received by the camera.  Usually in the case of raycast, for simplicity, once the ray hits an object in the 3d scene it stops and the color of the hit object is blended into that pixel.  \\

Instead of the ray cast ending after an arbitrary number of reflections, the rays’ direction is not affected by reflection.  Each perceptron in NeRF or Ellipsoid in GS is associated with an emission of light which adds its contribution to the ray by integrating the contribution of each perceptron or ellipsoid to the ray small changes to the light emitted are captured by the alpha-blending equation. Each of these are trained by taking training samples and finding the divergence between the rendering at the current training step compared to the known ground truth image.  In order, to have this information though you need to also know the pose of the camera relative to the scene during training which can be a major impediment to applying a radiance field method. In this case we have the benefit of a closed system where we can define the position of the viewer based on the geometry of the original imaging system which is published information.  \\

\subsection{Refinement}
The refinement module processes three inputs to construct a more refined mask. These three inputs are: the raw image volume, initial segmentation produced by the Cellpose~\cite{stringer2021cellpose} base model, and Gaussian splats constructed from the raw image volume. On the architecture side, this module shares the exact same architecture as CellpoSe~\cite{stringer2021cellpose}, except the number of input channels. Please refer to ~\Cref{fig:refinement}.

\begin{figure}[t]
    \centering
    \includegraphics[width=9cm]{figs/refinement.pdf}
    \caption{Refinement Strategy.}
    \label{fig:refinement}
\end{figure}

\subsection{Test-Time Adaptation}
\dots
\subsection{Cell Tracking}
\dots
\section{Experiments}
\subsection{Implementation}

We follow Cellpose~\cite{stringer2021cellpose} and utilize their U-Net architecture and the data to pretrain the network on a Titan Xp GPU for 500 iterations with a batch size of 8. Since our approach is considered a test-time adaptation approach, we adapt to test instances from the test split of various benchmarks we evaluate in this manuscript. We use SGD with the learning rate of 0.01, momentum 0.9, weight decay of 0.00001, and batch size of 2. We take square patches of size $h=w=112$, with a minimum overal of 48 during evaluation, and nominal cell size $m_n = 30$. Hyperparameters for adaptation losses are $\lambda_1 = \lambda_2 = 1.0$.  Following ~\cite{keaton2023celltranspose}, we set other hyperparameters as: $|\mathcal{N}_i|=20, \tau=0.1$, and $\theta=0.05$.

\subsection{Evaluation Metrics}
Following Cellpose~\cite{stringer2021cellpose}, we quantify our predictions by matching our predicted masks with the ground truth masks, and whichever provides the highest IoU becomes the pair we consider as true positive (TP). The masks without any valid matches are false positives (FP), and the ground truth masks without any valid matches are considered false negatives (FN). With these predictions defined, we compute the standard average precision metric AP for each image and then average it as defined below:
\begin{equation}
    AP = \frac{TP}{TP+FP+FN}
\end{equation}   
We also compute the F1 score which is given by:
\begin{equation}
    F1 = \frac{TP}{TP+\frac{1}{2}(FP+FN)}
\end{equation} 
\subsection{Datasets}
We choose Cellpose~\cite{stringer2021cellpose} Generalist dataset as our source data and \textbf{TissueNet}~\cite{TissueNet} data splits as our various domains we want to adapt to. TissueNet~\cite{TissueNet} collects tissue data (i.e., Breast, GI, Immune, Pancreas, etc.) from many platforms (i.e., CyCIF, CODEX, Vectra, MIBI, etc. ). We adapt to each of these specific domains separately in our experiments. In \Cref{tab:tn1,tab:tn2}, we show relative improvement over various dataset splits within TissueNet. The left portion describes the platform type and the right describes the tissue type. In each case, we show that we improve over the baseline. For example, for Codex-Pancreas, we improved from 0.717 to 0.742 in just one backward pass without any target labels or source data. We also present qualitative evaluation of our approach in \Cref{fig:qual}. In each of the cases illustrated, we can observe relatively low number of false positives and false negatives in the adapted version compared to the baseline.

\begin{table}
    
    \caption{Comparative analysis on various splits of TissueNet. We show improvement with test-time adaptation. }
    \label{tab:tn1}
    \scalebox{0.8}{
\begin{tabular}{c|cc|cc|cc}
    CP $\rightarrow$ TN  & \multicolumn{2}{l|}{Codex-Pancreas} & \multicolumn{2}{l|}{CyCIF-Immune} & \multicolumn{2}{l}{MIBI-Breast} \\
                            & {AP}             & F1                    & {AP}             & F1                    & {AP}             & F1                    \\ 
\hline
Cellpose                  & {0.717}          & 0.836                 & {0.476}          & 0.640                 & {0.228}          & 0.346                 \\ 
Cellpose-tta              & {\textbf{0.742}} & \textbf{0.853}        &{\textbf{0.501}} & \textbf{0.661}        & {\textbf{0.239}} & \textbf{0.361}        \\ 
CellTranspose             &                &  &               & & &  \\
\end{tabular}
    }
\end{table}

\begin{table}
    
    \caption{Comparative analysis on various splits of TissueNet. We show improvement with test-time adaptation. }
    \label{tab:tn2}
    \scalebox{0.8}{
\begin{tabular}{c|cc|cc|cc}
    CP $\rightarrow$ TN  & \multicolumn{2}{l|}{Mixif-GI} & \multicolumn{2}{l|}{Vectra-Immune} & \multicolumn{2}{l}{Vectra-Breast} \\
                            & {AP}             & F1                    & {AP}             & F1                    & {AP}             & F1                    \\ 
\hline
Cellpose                  & {0.355}          & 0.527                 & {0.629}          & 0.776                 & {0.574}          & 0.745                 \\ 
Cellpose-tta              & {\textbf{0.373}} & \textbf{0.545}        &{\textbf{0.631}} & \textbf{0.777}        & {\textbf{0.583}} & \textbf{0.751}        \\ 
CellTranspose             &                &  &               & & &  \\
\end{tabular}
    }
\end{table}


\begin{figure}
    \includegraphics[width=8.75cm]{figs/1.pdf}
    \includegraphics[width=8.75cm]{figs/43.pdf}
    \includegraphics[width=8.75cm]{figs/45.pdf}
    \includegraphics[width=8.75cm]{figs/55.pdf}
    \caption{\textbf{Illustating TTA results on Codex-Pancreas dataset from TissueNet against the ground truth and the baseline.} In each of the cases illustrated, we can observe relatively low number of false positives and false negatives in the adapted version compared to the baseline.}
    \label{fig:qual}
\end{figure}

\section{Conclusions}
We have presented our hypotheses and experiments for a Test-Time adaptation of a cell segmentation model called Cellpose~\cite{stringer2021cellpose}. We demonstrated the ability to train BN parameters through self-entropy and self-supervised, contrastive flow loss.  This was applied to a grid of domain adaptation tests from the Cellpose dataset to tissues from vastly different specimens and captured in a different imaging modality. We were able to show an increase over the baseline for all six of the experiments.   Due to the ubiquity of Batch Normalization in networks this approach can be applied in a lot more going forward.  For next steps towards making this a potentially successful paper we should be comparing this to other TTA approaches like those referenced in \ref{sec:RW}, until then we cannot position ourselves in the field accurately. 


{
    \small
    \bibliographystyle{ieeenat_fullname}
    \bibliography{main}
}

% WARNING: do not forget to delete the supplementary pages from your submission 
% \input{sec/X_suppl}

\end{document}
